\documentclass[11pt]{article}
\usepackage[hyphens]{url}
\usepackage[final,colorlinks=true,linkcolor=black,citecolor=black,urlcolor=blue]{hyperref}
\usepackage{verbatim}

% vertical spacing,
\voffset0in
\topmargin0in
\headheight0in
\headsep0in
\textheight9.0in
\footskip.4in

% horizontal spacing
\hoffset0in
\oddsidemargin0in
\evensidemargin0in
% \columnsep.1in (doesn't matter for NSF)
\marginparsep0in
\textwidth6.5in

\parindent 0in

% avoid widow lines
\widowpenalty=1000

\renewcommand{\baselinestretch}{1.0}

\newcommand{\hangpara}{
 \setlength{\parindent}{0in} % don't indent new paragraphs
 \hangindent=0.42in % indent all subsequent lines
}

\pagenumbering{arabic}

\begin{document}
\begin{center}
{\bf CURRICULUM VITAE\\Valeriia Perepelytsia}\\
\vskip 12pt






\begin{tabular}{ll}
Linguistic Research Infrastructure & valeriia.perepelytsia at uzh.ch \\
University of Zurich & \href{https://www.liri.uzh.ch/en/aboutus/Valeriia-Perepelytsia.html}{Webpage} \\ 
Andreasstrasse 15, 4.44 & \href{https://orcid.org/0009-0003-3355-9580}{ORCID iD} \\
8050 Zurich, Switzerland & \href{https://scholar.google.com/citations?user=9IiY_w8AAAAJ&hl=ru}{Google Scholar} \\
\end{tabular}
\end{center}





\vskip 12pt
\begin{flushleft}
{\bf EMPLOYMENT}
\end{flushleft}
\begin{tabular}{lp{5.5in}}
\bf 2019-- & {\bf University of Zurich, Switzerland} \\
& Postdoctoral researcher, \href{https://www.liri.uzh.ch/}{Linguistic Research Infrastructure }(current) \\
& Postdoctoral researcher, Department of Computational Linguistics (2024-2025) \\
& PhD student, Department of Computational Linguistics (2019-2024) \\
& Technical support for online examination, Department of Informatics (2021-2025) \\
& Student assistant at Phonetics and Speech Sciences group (2020-2021) \\
\bf 2024-- & {\bf JP French International, United Kingdom and Switzerland} \\
& Forensic speech and audio analyst (\href{https://www.jpfrench.com/valeriia-perepelytsia-4/}{Webpage}) \\
\bf 2018--2019 & {\bf Graz University of Technology, Austria} \\
& Student assistant, Signal Processing and Speech Communication Laboratory \\
\bf 2017--2018 & {\bf University of Graz, Austria} \\
& Student assistant, Department of English Studies \\
\end{tabular}





\vskip 12pt
\begin{flushleft}
{\bf EDUCATION}
\end{flushleft}
\begin{tabular}{lp{5.5in}}
\bf 2024 & University of Zurich, Switzerland \\
& \textbf{Ph.D.\ in Computational Linguistics and Phonetics \textit{(summa cum laude)}} \\
\bf 2018 & University of Graz, Austria \\
& \textbf{M.A.\ in English and American Studies (with honours)} \\
\bf 2014 & Kyiv National Linguistic University, Ukraine \\
& \textbf{B.A.\ in English Philology} \\
\end{tabular}


\vskip 12pt
\begin{flushleft}
{\bf RESEARCH MOBILITY}
\end{flushleft}
\begin{tabular}{lp{5.5in}}
\bf 2023 & University College London, Department of Speech, Hearing, and Phonetic Sciences \\
& Host: Prof. Dr. Carolyn McGettigan \\
\bf 2019 & Medical University of Graz, Gottfried Schatz Research Center for Cell Signaling, Metabolism and Aging; Host: Dr. Julia Feichtinger \\
\bf 2018 & University of Barcelona, Cognitive Biology of Language Group; \\
& Host: Prof. Dr. Cedric Boeckx \\
\bf 2016 & University of Bamberg, Institute for English and American Studies; \\ 
\end{tabular}



\vskip 18pt
\begin{flushleft}
{\bf JOURNAL PUBLICATIONS}
\end{flushleft}
\vskip 6pt
\hangpara
{\bf 2023}\hspace{1ex} \textbf{Perepelytsia, Valeriia} and Volker Dellwo. Acoustic compression in Zoom audio does not compromise voice recognition performance. \textit{Scientific Reports}. \url{https://doi.org/10.1038/s41598-023-45971-x}.




\vskip 18pt
\begin{flushleft}
{\bf CONFERENCE PROCEEDINGS (PEER-REVIEWED)}
\end{flushleft}
\vskip 6pt
\hangpara
{\bf 2023}\hspace{1ex} \textbf{Perepelytsia, Valeriia}, Leah Bradshaw and Volker Dellwo. IDEAR: A speech database of identity-marked, clear, and read speech. \textit{Proceedings of the 20th International Congress of Phonetic Sciences}, 3216-3220, Prague, Czech Republic. \url{https://doi.org/10.5167/uzh-236804}.
\vskip 6pt
\hangpara
{\bf 2023}\hspace{1ex} Leah, Bradshaw, \textbf{Valeriia Perepelytsia} and Volker Dellwo. Vocal effort in human interactions with voice-AI. \textit{Proceedings of the 20th International Congress of Phonetic Sciences}, 803-807, Prague, Czech Republic. \url{https://doi.org/10.5167/uzh-255338}.



\vskip 18pt
\begin{flushleft}
{\bf REPORTS}
\end{flushleft}
\vskip 6pt
\hangpara
{\bf 2024}\hspace{1ex} Prasad, A., Shahreza, H. O., Carofilis, A., Farhadipour, A., Liu, S., Madikeri, S., George, A., Motlicek, P., Marcel, S., Chapariniya, M., \textbf{Perepelytsia, V.}, Vukovic, T., Dellwo, V. \textit{Team Switzerland submission to NIST SRE24 speaker recognition evaluation.}
\vskip 6pt
\hangpara
{\bf 2024}\hspace{1ex} Farhadipour, A., Liu, S., Chapariniya, M., \textbf{Perepelytsia, V.}, Madikeri, S., Vukovic, T., Dellwo, V. (2024). \textit{CL-UZH submission to the NIST SRE 2024 speaker recognition evaluation.}



\vskip 20pt
\begin{flushleft}
{\bf PREPRINTS AND WORK IN PROGRESS}
\end{flushleft}
\vskip 6pt
\hangpara
\textbf{Perepelytsia, Valeriia}, Natalie Giroud, Martin Meyer and Volker Dellwo. Neural speech tracking is modulated by voice identity, but not by voice familiarity.{\it OSF Preprints}. \url{https://doi.org/10.31219/osf.io/kcwb2_v1}.
\vskip 6pt
\hangpara
\textbf{Perepelytsia, Valeriia}, Thayabaran Kathiresan, Guillaume Cordonnier, Elisa Pellegrino and Volker Dellwo. SEE-U-VR: A novel method for eliciting different speaking styles  using virtual reality. In E. Glaser, J. Kabatek, B. Sonnenhauser (Eds.): \textit{Sprachenräume der Schweiz. Band 2.}






\vskip 20pt
\begin{flushleft}
{\bf INVITED PRESENTATIONS}
\end{flushleft}

\hangpara
{\bf 2024}\hspace{1ex}“Voice individuality and recognition – novel approaches and future prospects”. Neural Bases of Communication (BaNCo) lab, Aix-Marseille University. December 12. Marseille, France. 

\vskip 6pt
\hangpara
{\bf 2024}\hspace{1ex}“Neural correlates speech and voice processing in older adults”. Language and Medicine Colloquium, University of Zurich. November 7. Zurich, Switzerland. 

\vskip 6pt
\hangpara
{\bf 2023}\hspace{1ex}“Voice perception in younger and older adults”. Institute of Biology, University of Neuchâtel. May 25. Neuchâtel, Switzerland. 

\vskip 6pt
\hangpara
{\bf 2023}\hspace{1ex} “The effect of channel quality and listeners’ age on voice perception”. Speech Science Forum, Department of Speech, Hearing, and Phonetic Sciences, University College London. March 2. London, UK.

\vskip 6pt
\hangpara
{\bf 2023}\hspace{1ex} “Identity-marked speech as a strategy to enhance voice recognizability”. Forensic Speech Science Research Group, University of York. February 8. York, UK. Co-presenter with Leah Bradshaw.

\vskip 6pt
\hangpara
{\bf 2023}\hspace{1ex} “Experimental methods in speech production”. Presenter in the course “Brain, language, experiments” by Prof. Dr. Martin Meyer, University of Zurich. May 25. Zurich, Switzerland. 

\vskip 6pt
\hangpara
{\bf 2023}\hspace{1ex} “Building online experiments in Gorilla”. Workshop by Linguistic Research Infrastructure (LiRI), University of Zurich. March 15. Zurich, Switzerland. Co-presenter with Leah Bradshaw.

\vskip 6pt
\hangpara
{\bf 2022}\hspace{1ex} “Identity-marked speech as a strategy to enhance voice recognizability”. Slavisches Seminar, University of Zurich. December 9. Zurich, Switzerland.

\vskip 6pt
\hangpara
{\bf 2021}\hspace{1ex} “Talker familiarity advantage in younger and older adults”. Colloquium Phonetics and Speech Sciences; University of Zurich. December 8. Zurich, Switzerland.

\vskip 6pt
\hangpara
{\bf 2021}\hspace{1ex} “Speech perception in older adults”. URPP Dynamics of Healthy Aging; University of Zurich. December 7. Zurich, Switzerland.

\vskip 6pt
\hangpara
{\bf 2021}\hspace{1ex} “Talker familiarity effect in younger adults”. Colloquium Neuroscience of Speech and Language; University of Zurich. November 25. Zurich, Switzerland.

\vskip 6pt
\hangpara
{\bf 2021}\hspace{1ex} “Neural signatures of voice processing”. Colloquium of the Department of Comparative Language Science; University of Zurich. October 10. Zurich, Switzerland.

\vskip 6pt
\hangpara
{\bf 2021}\hspace{1ex} “Neural signatures of voice identity: an EEG study of familiar voice processing”. Colloquium Neuroscience of Speech and Language; University of Zurich. May 20. Zurich, Switzerland.

\vskip 6pt
\hangpara
{\bf 2020}\hspace{1ex} “Talker familiarity effect and speech envelope tracking”. Colloquium Neuroscience of Speech and Language; University of Zurich. October 10. Zurich, Switzerland.

\vskip 6pt
\hangpara
{\bf 2020}\hspace{1ex} “The emergence of language-ready brain”. Department of English Studies, University of Graz. May 10. Graz, Austria.

\vskip 6pt
\hangpara
{\bf 2020}\hspace{1ex} “The emergence of Merge”. Department of English Studies, University of Graz. April 7. Graz, Austria.





\vskip 20pt
\begin{flushleft}
{\bf CONFERENCE PRESENTATIONS}
\end{flushleft}

\vskip 6pt
\hangpara
{\bf 2024}\hspace{1ex}“Neural correlates of voice familiarity and identity: an EEG study”. Poster presented at the 2nd Interdisciplinary Conference on Voice Identity (VoiceID). August 28--30. Marburg, Germany.


\vskip 6pt
\hangpara
{\bf 2023}\hspace{1ex}“IDEAR: A speech database of identity-marked, clear, and read speech”. Paper presented at the 20th International Congress of Phonetic Sciences. August 7--11. Prague, Czech Republic.


\vskip 6pt
\hangpara
{\bf 2023}\hspace{1ex}“Own-age bias in voice recognition by younger and older adults”. Paper presented at the 31st Annual Conference of the International Association for Forensic Phonetics and Acoustics. July 9--12. Zurich, Switzerland.


\vskip 6pt
\hangpara
{\bf 2023}\hspace{1ex}“Investigating an own-age bias in voice recognition”. Paper presented at the VI International Round Table “Current Trends in Phonetic Studies”. April 23. Kyiv, Ukraine (online).


\vskip 6pt
\hangpara
{\bf 2022}\hspace{1ex}“Neural underpinnings of familiar talker advantage: an EEG study”. Paper presented at the 30th Annual Conference of the International Association for Forensic Phonetics and Acoustics. July 10--13. Prague, Czech Republic.


\vskip 6pt
\hangpara
{\bf 2022}\hspace{1ex}“Can voice recognizability be controlled by speakers? A study on identity marked speech”. Poster presented at the 1st Interdisciplinary Conference on Voice Identity (VoiceID): Perception, Production, and Computational Approaches. July 4--6. Zurich, Switzerland.


\vskip 6pt
\hangpara
{\bf 2022}\hspace{1ex}“Talker familiarity effect: exploring individual differences in voice learning”. Poster presented at the Language and Medicine Market. June 20. Zurich, Switzerland.


\vskip 6pt
\hangpara
{\bf 2021}\hspace{1ex}“Does audio recording through video-conferencing tools hinder voice recognition performance? A comparison study on different audio channel recordings”. Paper presented at the IAFPA2021: The Anual Conference of the International Association for Forensic Phonetics and Acoustics. August 22-25. Marburg, Germany (online).


\vskip 6pt
\hangpara
{\bf 2017}\hspace{1ex}“Singing talent and second language pronunciation”. Paper presented at Österreichische Linguistiktagung. December 8-10. Klagenfurt, Austria.


\vskip 6pt
\hangpara
{\bf 2017}\hspace{1ex}“Homo Loquens: a comprehensive account of language”. Paper presented at the 10th Annual Embodied and Situated Language Processing Conference. September 10-12. Moscow, Russia.


\vskip 6pt
\hangpara
{\bf 2017}\hspace{1ex}“Homo Loquens: a comprehensive account of language”. Poster presented at the Model-Based Cognitive Neuroscience summer school. 31 July - 04 August. Amsterdam, the Netherlands.


\vskip 6pt
\hangpara
{\bf 2017}\hspace{1ex}“Hunting genes for language”. Paper presented at Spanning Regions and Disciplines: 2nd Student Conference of the European Joint Master's Degree in English and American Studies. June 22. Bamberg, Germany.


\vskip 6pt
\hangpara
{\bf 2014}\hspace{1ex}“Neural properties of poetic imagery: A case study of W. Shakespeare’ sonnets”. Paper presented at the Language, Literature, Works of Art – Cognitive-Semiotic Interface. September 25--27. Kyiv, Ukraine.


\vskip 20pt
\begin{flushleft}
{\bf OUTREACH / MEDIA}
\end{flushleft}

\hangpara
{\bf 2022}\hspace{1ex}PhonPod Podcast - Episode 5 - Valeriia Perepelytsia. \url{https://open.spotify.com/episode/2s04a7rksrNl7rez7NF9RY?si=724be79d09e340f4}.



\vskip 20pt
\begin{flushleft}
{\bf TEACHING EXPERIENCE}
\end{flushleft}
\begin{flushleft}
{\bf University courses (co-instructor)}
\end{flushleft}
\hangpara Evolutionary neuroscience of language. University of Zurich, annually Fall 2023--2025.
\vskip 6pt
\hangpara Speech perception and the brain. University of Zurich, Fall 2023--2024.
\vskip 6pt
\hangpara Neurobiologische Theorien. University of Zurich, Spring 2021.
\hangpara Academic writing. University of Graz, Spring 2017.




\begin{flushleft}
{\bf Summer schools (co-instructor)}
\end{flushleft}
\hangpara Using Gorilla Experiment Builder for perception experiments. Summer School “Experimental methods in vocal identity research. Production, perception and acoustic modelling of human and animal vocalizations”. Department of Computational Linguistics, University of Zurich. 4-13 September 2023.




\vskip 20pt
\begin{flushleft}
{\bf ADVISING}
\end{flushleft}
\hangpara \textbf{Nataliya Fartdinova}, Linguistics (focus area "Phonetics and Speech Sciences"). 2023. M.A. co-advisor. Thesis: “Talker familiarity effect on speech perception in older adults”. University of Zurich, Switzerland.
\vskip 6pt
\hangpara \textbf{Tugce Aras}, Psychology. 2022. M.A. co-advisor. Thesis: “Neural correlates of speaker identity processing in adverse listening conditions”. University of Zurich, Switzerland.




\vskip 20pt
\begin{flushleft}
{\bf AWARDS}
\end{flushleft}
\hangpara
{\bf 2022}\hspace{1ex}IAFPA Student Prize for the best oral presentation at IAFPA 2022 conference, Prague, Czech Republic.
\vskip 6pt
\hangpara
{\bf 2022}\hspace{1ex}Oxford Wave Research prize for the best automatic- or audio-focussed paper at IAFPA 2022 conference; Prague, Czech Republic. %(£ 250)




\vskip 20pt
\begin{flushleft}
{\bf GRANTS}
\end{flushleft}
\hangpara
{\bf 2023}\hspace{1ex}Research stay grant from the Graduate School of the Faculty of Arts and Social Sciences, University of Zurich, Switzerland. %(CHF 2122)
\vskip 6pt
\hangpara
{\bf 2019-2024}\hspace{1ex}Travel grants for conference travel, University of Zurich, Switzerland. %(CHF 2122)




\vskip 20pt
\begin{flushleft}
{\bf ACADEMIC SERVICE: CONFERENCE ORGANIZATION}
\end{flushleft}
\hangpara
{\bf 2023}\hspace{1ex} 31st conference of the International Association for Forensic Phonetics and Acoustics (IAFPA), co-organized with Leah Bradshaw, Volker Dellwo, Alessandro De Luca, Peter French, Daniel Friedrichs, Andrea Frölich, Lei He, Carolina Lins Machado, Elisa Pellegrino, and Alejandra Pesantez. July 9-12, 2023, Zurich, Switzerland.


\vskip 20pt
\begin{flushleft}
{\bf ACADEMIC SERVICE: CONFERENCE REVIEWING}
\end{flushleft}
\hangpara
{\bf 2023}\hspace{1ex} Reviewer, 31st conference of the International Association for Forensic Phonetics and Acoustics (IAFPA).




\vskip 20pt
\begin{flushleft}
{\bf ACADEMIC SERVICE: UNIVERSITY SERVICE}
\end{flushleft}
\hangpara
{\bf 2020-2024}\hspace{1ex} Representative of the junior researchers of the Department of Computational Linguistics, University of Zurich.




\vskip 20pt
\begin{flushleft}
{\bf ACADEMIC SERVICE: JOURNAL REVIEWING}
\end{flushleft}
\hangpara
{\bf 2025}\hspace{1ex}Reviewer for \textit{Forensic Science International}.
\vskip 6pt
\hangpara
{\bf 2024}\hspace{1ex}Reviewer for \textit{NeuroImage}.
\vskip 6pt
\hangpara
{\bf 2024}\hspace{1ex}Reviewer for \textit{JMIR Biomedical Engineering}.



\vskip 20pt
\begin{flushleft}
{\bf PROFESSIONAL MEMBERSHIPS (PAST \& PRESENT)}
\end{flushleft}
\begin{itemize}
\item International Association for Forensic Phonetics and Acoustics (IAFPA)
\item Linguistic Society of America (LSA)
\item Ukrainian Association for Cognitive Linguistics and Poetics (UACLiP)
\end{itemize}



%\clearpage
\vskip 10pt
\begin{flushleft}
{\bf LANGUAGES}
\end{flushleft}
\begin{itemize}
\item {\bf Russian} native
\item {\bf Ukrainian} native
\item {\bf English} proficient
\item {\bf German} proficient
\item {\bf French} intermediate
\end{itemize}

\vskip 20pt
\today
\end{document}
